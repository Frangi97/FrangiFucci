\documentclass[12pt,a4paper]{report}
\title{SafeStreets - RASD \\ \large version 1.0}
\author{Frangi Alberto, Fucci Tiziano}
\date{A.Y. 2019/2020}
\begin{document}
\maketitle
\tableofcontents
\chapter{Introduction}
	\section{Purpose}
		\subsection{General Purpose}
SafeStreets is a crowd-sourced application that intends to provide users with the possibility to notify authorities when traffic violations occur, and in particular parking violations. The application allows users to send pictures of violations, including their date, time, and position, to authorities. Examples of violations are vehicles parked in the middle of bike lanes or in places reserved for people with disabilities, double parking, and so on.

SafeStreets stores the information provided by users, completing it with suitable meta-data. In particular, when it receives a picture, it runs an algorithm to read the license plate (one can also think of mechanisms with which the user can help with the recognition), and stores the retrieved information with theviolation, including also the type of the violation (input by the user) and the name of the street where the violation occurred (which can be retrieved from the geographical position of the violation). 

The application allows both end users and authorities to mine the information that has been received, for example by highlighting the streets (or the areas) with the highest frequency of violations, or the vehicles that commit the most violations. Of course, different levels of visibility could be offered to different roles.

Moreover, if the municipality offers a service that allows users to retrieve the information about the accidents that occur on the territory of the municipality, SafeStrees can cross this information with its own data to identify potentially unsafe areas, and suggest possible interventions (e.g., add a barrier between the bike lane and the part of the road for motorized vehicles to prevent unsafe parking).

		\subsection{Goals}
	\section{Scope}
	\section{Definitions, acronyms, abbreviations}
	\section{Revision history}
	\section{Reference documents}
	\section{Document structure}
%end first chapter

\chapter{Overall description}
	\section{Product perspective}
	\section{Product functions}
	\section{User characteristics}
	\section{Assumptions, dependencies and constraints}
%end second chapter

\chapter{Specific Requirments}
	\section{External interface requirments}
		\subsection{User interfaces}
		\subsection{Hardware interfaces}
		\subsection{Software interfaces}
		\subsection{Communication interfaces}
	\section{Functional Requirements}
	\section{Performances Requirments}
	\section{Design constraints}
		\subsection{Standards compliance}
		\subsection{Hardware limitations}
		\subsection{Any other costraint}
	\section{Software System Attributes}
		\subsection{Reliability}
		\subsection{Availability}
		\subsection{Security}
		\subsection{Maintainability}
		\subsection{Portablity}
%end third chapter

\chapter{Formal Analysis}
%end fourth chapter

\chapter{Effort Spent}
%end fiveth chapter

\chapter{References}
%end sixth chapter
\end{document}